
% LaTeX Beamer file automatically generated from DocOnce
% https://github.com/hplgit/doconce

%-------------------- begin beamer-specific preamble ----------------------

\documentclass{beamer}

\usetheme{red_plain}
\usecolortheme{default}

% turn off the almost invisible, yet disturbing, navigation symbols:
\setbeamertemplate{navigation symbols}{}

% Examples on customization:
%\usecolortheme[named=RawSienna]{structure}
%\usetheme[height=7mm]{Rochester}
%\setbeamerfont{frametitle}{family=\rmfamily,shape=\itshape}
%\setbeamertemplate{items}[ball]
%\setbeamertemplate{blocks}[rounded][shadow=true]
%\useoutertheme{infolines}
%
%\usefonttheme{}
%\useinntertheme{}
%
%\setbeameroption{show notes}
%\setbeameroption{show notes on second screen=right}

% fine for B/W printing:
%\usecolortheme{seahorse}

\usepackage{pgf,pgfarrows,pgfnodes,pgfautomata,pgfheaps,pgfshade}
\usepackage{graphicx}
\usepackage{epsfig}
\usepackage{relsize}

\usepackage{fancybox}  % make sure fancybox is loaded before fancyvrb

\usepackage{fancyvrb}
%\usepackage{minted} % requires pygments and latex -shell-escape filename
%\usepackage{anslistings}

\usepackage{amsmath,amssymb,bm}
%\usepackage[latin1]{inputenc}
\usepackage[T1]{fontenc}
\usepackage[utf8]{inputenc}
\usepackage{colortbl}
\usepackage[english]{babel}
\usepackage{tikz}
\usepackage{framed}
% Use some nice templates
\beamertemplatetransparentcovereddynamic

% --- begin table of contents based on sections ---
% Delete this, if you do not want the table of contents to pop up at
% the beginning of each section:
% (Only section headings can enter the table of contents in Beamer
% slides generated from DocOnce source, while subsections are used
% for the title in ordinary slides.)
\AtBeginSection[]
{
  \begin{frame}<beamer>[plain]
  \frametitle{}
  %\frametitle{Outline}
  \tableofcontents[currentsection]
  \end{frame}
}
% --- end table of contents based on sections ---

% If you wish to uncover everything in a step-wise fashion, uncomment
% the following command:

%\beamerdefaultoverlayspecification{<+->}

\newcommand{\shortinlinecomment}[3]{\note{\textbf{#1}: #2}}
\newcommand{\longinlinecomment}[3]{\shortinlinecomment{#1}{#2}{#3}}

\definecolor{linkcolor}{rgb}{0,0,0.4}
\hypersetup{
    colorlinks=true,
    linkcolor=linkcolor,
    urlcolor=linkcolor,
    pdfmenubar=true,
    pdftoolbar=true,
    bookmarksdepth=3
    }
\setlength{\parskip}{0pt}  % {1em}

\newenvironment{doconceexercise}{}{}
\newcounter{doconceexercisecounter}
\newenvironment{doconce:movie}{}{}
\newcounter{doconce:movie:counter}

\newcommand{\subex}[1]{\noindent\textbf{#1}}  % for subexercises: a), b), etc

%-------------------- end beamer-specific preamble ----------------------

% Add user's preamble




% insert custom LaTeX commands...

\raggedbottom
\makeindex

%-------------------- end preamble ----------------------

\begin{document}




% ------------------- main content ----------------------

% Text for ICNT workshop


% ----------------- title -------------------------

\title{International Collaborations in Nuclear Theory: Theory for open-shell nuclei near the limits of stability}

% ----------------- author(s) -------------------------

\author{Scott K. Bogner (NSCL/Michigan State University), Morten Hjorth-Jensen (NSCL/Michigan State University) and Jason D. Holt (TRIUMF)\inst{}}
\institute{}
% ----------------- end author(s) -------------------------


\date{May 11-29, 2015, Michigan State University and FRIB/NSCL
% <optional titlepage figure>
}

\begin{frame}[plain,fragile]
\titlepage
\end{frame}

\begin{frame}[plain,fragile]
\frametitle{Motivation}

\begin{block}{}


The pioneering activities of rare-isotope beam (RIB) facilities worldwide have 
ushered in a new era of nuclear physics over the past decades. With the advent of 
the next generation of facility upgrades, such as FRIB in the US, FAIR in Germany, 
and RIBF in Japan, thousands of undiscovered nuclei, often existing at the very 
limits of stability, will be created and studied in the laboratory. These exotic 
systems exhibit behavior distinct from their stable counterparts, and the quest to 
discover and understand their properties from first principles represents a 
cornerstone of modern nuclear science.

Since much of nuclear structure theory has been developed in the
context of stable systems, one of the central theoretical challenges
in the next few years will be to develop new or extend existing models
to elucidate and predict the novel features of nuclei at the limits of
stability.  

At the heart of this theoretical effort are three fundamental issues that must be confronted:
\begin{itemize}
\item the effects of many-nucleon, in particular three-nucleon (3N), forces, 

\item the development of reliable techniques for solving the nuclear many-body problem in the medium- and heavy- mass regions of the nuclear chart with quantifiable theoretical uncertainties, and 

\item the proper inclusion of continuum degrees of freedom for loosely bound systems. 
\end{itemize}

\noindent
The primary objective of this program will be to develop theoretical tools to  address these issues head-on, thereby laying the groundwork for a predictive and comprehensive theory of open-shell nuclei at the limits of stability.  The emerging predictions can furthermore serve as a guide for future experimental efforts at RIB facilities, such as FRIB at Michigan State University, where we propose to host this program.  Therefore the contributions of leading experimental scientists exploring exotic regions that link tightly with this proposal such as oxygen, fluorine, neon, potassium, calcium, nickel, copper and tin isotopes, will constitute an important component of this program.
\end{block}
\end{frame}

\begin{frame}[plain,fragile]
\frametitle{Plans for the workshop}

\begin{block}{}
The workshop aims at addressing three important theoretical topics in the analysis
of present and future experiments.  These are
\begin{itemize}
\item Construction of effective Hamiltonians for nuclei close to the driplines

\item Advances in large-scale computations of many-body systems

\item Applications to nuclear physics observables
\end{itemize}

\noindent
The first topic of the program will focus on constructing non-empirical 
valence-space Hamiltonians and effective operators using \emph{ab initio} 
many-body methods starting from the underlying two- and three-body interactions 
between nucleons. We will explore how this can be achieved within different  \emph{ab initio} frameworks, and the potential to use the resulting non-empirical
valence-space Hamiltonians as a possible benchmark between many-body 
theories. We will also examine how such non-empirical valence Hamiltonians can 
serve as an improved starting point for the development of empirical shell model 
interactions adjusted to data. 

The workshop will also focus on advancements in large-space diagonalization techniques which will be needed to efficiently 
deal with the large dimensions encountered in mid-shell systems with explicit treatment of three-body interactions and continuum degrees of freedom. Applications and confrontation with experiment is an equally essential ingredient of this workshop.
At the end of the program, a small group of core participants will generate a roadmap that lays out the necessary next steps towards the development of a predictive and comprehensive theory of exotic open-shell nuclei (e.g., identify near-term benchmark calculations and key experimental observables to constrain/test theory, assess the strengths and weaknesses of the methods/algorithms discussed during the program, etc.). This roadmap, together with a summary of the present status of the field, will be written up in a final report. Additionally, all presentations will be publicly available at the program web site. 
\end{block}
\end{frame}

\begin{frame}[plain,fragile]
\frametitle{Scientific goals}

\begin{block}{1) Construction of effective Hamiltonians for nuclei close to the driplines }
The framework of many-body perturbation theory (MBPT) has been used in the past 
for constructing effective valence-space Hamiltonians, but it is known to suffer a number of shortcomings (issues of convergence, sensitivity to harmonic oscillator frequency, etc.), and extending it to nuclei near the drip lines requires numerous non-trivial extensions. While MBPT has previously been used calculate 
valence-shell interactions for medium-mass nuclei based on NN+3N forces, 
recently a number of improved non-perturbative methods have been developed for this purpose.  A prescription exists to construct a valence-space Hamiltonian from no-core shell model 
calculations but is currently limited to $p$-shell nuclei.  The first 
nonperturbative technique for constructing valence-space Hamiltonians in 
medium-mass nuclei that is computationally feasible is the in-medium similarity 
renormalization group (IM-SRG), which decouples orbitals outside a given 
valence space through a continuous sequence of unitary transformations.  
We will investigate applications of the IM-SRG and MBPT (in standard and extended valence 
spaces) in the exotic oxygen and fluorine isotopes to evaluate the performance of both methods while highlighting possible new experiments which could further constrain and test both methods.  A particular focus will be on cross-shell matrix elements, which may be prone to spurious center-of-mass contamination and are found to 
be large in MBPT calculations, but are widely set to small values in 
phenomenological studies.  We will also discuss other approaches to construct valence Hamiltonians, for example from 
coupled-cluster theory and self-consistent Green's function methods. Finally, we will examine how the non-empirical valence Hamiltonians can be used as an improved starting point for constructing empirical shell model interactions adjusted to data.
\end{block}

\begin{block}{2) Large-scale diagonalization and improving phenomenological models  }
Exact diagonalization of extended valence-space Hamiltonians with continuum degrees of freedom quickly becomes a bottleneck due to the combinatorial growth in
dimensions encountered for mid-shell nuclei. Methods such as Monte Carlo shell model
and importance-truncated shell model, however, present attractive
alternatives to exact diagonalization in extended valence spaces.  The
recently developed full configuration interaction quantum Monte Carlo method (FCIQMC), holds great promise for tackling systems with large
spaces, making it possible to study nuclei close to the limits of
stability.

Ultimately our theoretical understanding of most medium-mass nuclei has come 
from phenomenological models, which take various microscopic valence-space 
Hamiltonians as starting points, which are then adjusted to experimental data.  Since in some cases valence-space Hamiltonians based on NN+3N forces can describe experimental 
data at a level comparable with phenomenology, it is an open question whether a 
common mechanism exists between models which is responsible for exotic 
phenomena such as new magic numbers.  We can furthermore explore the 
intriguing possibility to produce new phenomenological Hamiltonians starting from 
the improved microscopic Hamiltonians in both standard and 
extended valence spaces. The latter should most likely include effects from resonant states and/or the non-resonant continuum.
\end{block}

\begin{block}{3) Applications to nuclear physics observables }

Besides the theoretical developments which are needed to tackle systems with many degrees of freedom, the theory developments  will be tightly linked with key experimental results and programs of nuclei close to the limit of stability. 
To confront theory with data is essential in developing reliable theoretical tools. In particular, for coming studies 
of weak interactions in nuclei.

\end{block}
\end{frame}

\begin{frame}[plain,fragile]
\frametitle{Possible additional topics}

\begin{block}{Developing theoretical uncertainties }
Global theoretical
studies of isotopic chains, such as the chains of calcium, nickel or
tin isotopes, make it possible to test systematic properties of
nuclear models, although a quantitative
comparison of various experimental data with quantified theoretical
uncertainties still remains a major challenge for nuclear science. To
address this shortcoming, we would also like to 
focus on developing reliable theoretical error bars in many-body
calculations of open-shell nuclei. This is a multifaceted endeavor that will involve examining the interplay of uncertainties from various sources
such as effective field theory (EFT) truncation errors and uncertainties in the
fitted parameters of the input EFT interactions, truncated renormalization, basis-set
truncation errors, and truncation errors in the particular level of
many-body approximation. 
\end{block}
\end{frame}

\begin{frame}[plain,fragile]
\frametitle{Contact}

\begin{block}{}
Please contact any of the organizers (bogner@nscl.msu.edu, hjensen@msu.edu, holt@triumf.ca) or Linna Leslie with any questions, comments, or concerns:
lesliel@nscl.msu.edu Phone +1-(517) 908-7333, Fax +1-(517) 353-5967 

\end{block}
\end{frame}

\end{document}
