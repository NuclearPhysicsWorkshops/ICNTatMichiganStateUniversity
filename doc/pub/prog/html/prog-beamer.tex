
% LaTeX Beamer file automatically generated from DocOnce
% https://github.com/hplgit/doconce

%-------------------- begin beamer-specific preamble ----------------------

\documentclass{beamer}

\usetheme{red_plain}
\usecolortheme{default}

% turn off the almost invisible, yet disturbing, navigation symbols:
\setbeamertemplate{navigation symbols}{}

% Examples on customization:
%\usecolortheme[named=RawSienna]{structure}
%\usetheme[height=7mm]{Rochester}
%\setbeamerfont{frametitle}{family=\rmfamily,shape=\itshape}
%\setbeamertemplate{items}[ball]
%\setbeamertemplate{blocks}[rounded][shadow=true]
%\useoutertheme{infolines}
%
%\usefonttheme{}
%\useinntertheme{}
%
%\setbeameroption{show notes}
%\setbeameroption{show notes on second screen=right}

% fine for B/W printing:
%\usecolortheme{seahorse}

\usepackage{pgf,pgfarrows,pgfnodes,pgfautomata,pgfheaps,pgfshade}
\usepackage{graphicx}
\usepackage{epsfig}
\usepackage{relsize}

\usepackage{fancybox}  % make sure fancybox is loaded before fancyvrb

\usepackage{fancyvrb}
%\usepackage{minted} % requires pygments and latex -shell-escape filename
%\usepackage{anslistings}
%\usepackage{listingsutf8}

\usepackage{amsmath,amssymb,bm}
%\usepackage[latin1]{inputenc}
\usepackage[T1]{fontenc}
\usepackage[utf8]{inputenc}
\usepackage{colortbl}
\usepackage[english]{babel}
\usepackage{tikz}
\usepackage{framed}
% Use some nice templates
\beamertemplatetransparentcovereddynamic

% --- begin table of contents based on sections ---
% Delete this, if you do not want the table of contents to pop up at
% the beginning of each section:
% (Only section headings can enter the table of contents in Beamer
% slides generated from DocOnce source, while subsections are used
% for the title in ordinary slides.)
\AtBeginSection[]
{
  \begin{frame}<beamer>[plain]
  \frametitle{}
  %\frametitle{Outline}
  \tableofcontents[currentsection]
  \end{frame}
}
% --- end table of contents based on sections ---

% If you wish to uncover everything in a step-wise fashion, uncomment
% the following command:

%\beamerdefaultoverlayspecification{<+->}

\newcommand{\shortinlinecomment}[3]{\note{\textbf{#1}: #2}}
\newcommand{\longinlinecomment}[3]{\shortinlinecomment{#1}{#2}{#3}}

\definecolor{linkcolor}{rgb}{0,0,0.4}
\hypersetup{
    colorlinks=true,
    linkcolor=linkcolor,
    urlcolor=linkcolor,
    pdfmenubar=true,
    pdftoolbar=true,
    bookmarksdepth=3
    }
\setlength{\parskip}{0pt}  % {1em}

\newenvironment{doconceexercise}{}{}
\newcounter{doconceexercisecounter}
\newenvironment{doconce:movie}{}{}
\newcounter{doconce:movie:counter}

\newcommand{\subex}[1]{\noindent\textbf{#1}}  % for subexercises: a), b), etc

%-------------------- end beamer-specific preamble ----------------------

% Add user's preamble




% insert custom LaTeX commands...

\raggedbottom
\makeindex

%-------------------- end preamble ----------------------

\begin{document}

% endif for #ifdef PREAMBLE



% ------------------- main content ----------------------

% Text for ICNT workshop


% ----------------- title -------------------------

\title{International Collaborations in Nuclear Theory: Theory for open-shell nuclei near the limits of stability}

% ----------------- author(s) -------------------------

\author{\href{{http://nscl.msu.edu/directory/bogner.html}}{Scott K. Bogner} (NSCL/Michigan State University), \href{{https://www.pa.msu.edu/profile/hjensen}}{Morten Hjorth-Jensen} (NSCL/Michigan State University) and \href{{http://www.triumf.ca/jason-d-holt}}{Jason D. Holt} (TRIUMF)\inst{}}
\institute{}
% ----------------- end author(s) -------------------------


\date{May 11-29, 2015, Michigan State University and FRIB/NSCL
% <optional titlepage figure>
}

\begin{frame}[plain,fragile]
\titlepage
\end{frame}

\begin{frame}[plain,fragile]
\frametitle{Venue: Lecture Hall at the National Superconducting Cyclotron Laboratory}

\begin{block}{}
We meet all at the \textbf{Theory Trailer} (Trailer nr. 7) on the eastern side of the \href{{http://www.nscl.msu.edu/}}{National Superconducting Cyclotron
Laboratory} (640 South Shaw Lane on the MSU campus).  
We meet at 8.30am at the \textbf{Theory Trailer} for registration. Coffee plus light refreshments are served. Thereafter we walk over to the NSCL Lecture Hall. Lectures start at 9am all days.
The Theory Trailer is located close the construction site of the new NSCL/FRIB building (you will see a series of trailers). The Theory Trailer is the one closest to the \href{{http://www.whartoncenter.com}}{Wharton Center for performing arts at MSU} and Bogue street. 
From the Town Place hotel it is a walk of at \href{{https://www.google.com/maps/dir/2855+Hannah+Blvd,+East+Lansing,+MI+48823/640+South+Shaw+Lane,+East+Lansing,+MI/@42.7219308,-84.4747951,15z/data=!3m1!4b1!4m14!4m13!1m5!1m1!1s0x8822c2bce3547475:0xd5a97d9b6a59e537!2m2!1d-84.456337!2d42.718323!1m5!1m1!1s0x8822c2841754be71:0xb0ff5dca5dac1c71!2m2!1d-84.4738287!2d42.7247813!3e2}}{most 20-25 min}.

For those of you arriving by plane, if Detroit Metro is your final airport, we recommend using the \href{{http://www.michiganflyer.com}}{Michigan Flyer} bus service to East Lansing. Take thereafter a taxi to the Town Place hotel. If you are staying at the Kellogg conference center, they organize a shuttle transport from downtown East Lansing. If Lansing is your final airport, we recommend taking a taxi directly to the hotel. Alternatively, you can use CATA bus line 14 and connect with bus line 1 afterwards. 
\end{block}
\end{frame}

\begin{frame}[plain,fragile]
\frametitle{Recording of lectures and live streaming}

\begin{block}{}
For those of you who cannot attend the lectures, these are streamed live (access only for people at the NCSL). The link to the website is \href{{https://arachnid.nscl.msu.edu/video/icnt/}}{\nolinkurl{https://arachnid.nscl.msu.edu/video/icnt/}}
\end{block}
\end{frame}

\begin{frame}[plain,fragile]
\frametitle{Program first week May 11-15 2015}

\begin{block}{}


{\footnotesize
\begin{tabular}{cccccc}
\hline
\multicolumn{1}{c}{ Time } & \multicolumn{1}{c}{ Monday } & \multicolumn{1}{c}{ Tuesday } & \multicolumn{1}{c}{ Wednesday } & \multicolumn{1}{c}{ Thursday } & \multicolumn{1}{c}{ Friday } \\
\hline
9am-10am                                                                            & \href{{https://scholar.google.com/citations?user=FpcHIs8AAAAJ&hl=en}}{Andreas Ekstrom} (ORNL/UTK) & \href{{http://tkynt2.phys.s.u-tokyo.ac.jp/~otsuka/index.html.en}}{Takaharu Otsuka} (Tokyo) & \href{{https://www.pa.msu.edu/profile/spyrou}}{Artemisia Spyrou} (MSU)                                      & \href{{https://scholar.google.com/citations?user=AhFzlysAAAAJ&hl=en}}{Gustav Jansen} (UTK/ORNL)    & \href{{http://francesa.phy.cmich.edu/people//horoi/}}{Mihai Horoi} (CMU)   \\
Clustering, Shapes and Nuclear Transmutation studied by the Monte Carlo Shell Model & Constraining the Description of the Nuclear Interaction                                           &                                                                                            & Using beta-decays to constrain (n,g) reaction cross sections on short lived nuclei                          & Effective interactions and operators from coupled-cluster theory                                   & Nuclear structure for tests of fundamental symmetries                      \\
\hline
10.0am-10.30am                                                                      & Coffee break                                                                                      & Coffee break                                                                               & Coffee break                                                                                                & Coffee break                                                                                       & Coffee break                                                               \\
\hline
10.30am-11.30am                                                                     & \href{{https://plus.google.com/105561779205361798723/about}}{Heiko Hergert} (MSU)                 & \href{{https://people.nscl.msu.edu/~gade/}}{Alexandra Gade} (MSU)                          & \href{{https://scholar.google.com/citations?user=IWqpG7EAAAAJ}}{Yutaka Utsuno} (Japan Atomic Energy Agency) & \href{{http://nuclear-structure.lbl.gov/people/crawford}}{Heather Crawford} (LBNL)                 & \href{{http://www.physics.uoguelph.ca/~gezerlis/}}{Alex Gezerlis} (Guelph) \\
                                                                                    & Effective interactions and operators from IM-SRG                                                  & Nuclear spectroscopy of exotic nuclei                                                      & Probing shell evolution with large-scale shell-model calculations                                           & Neutron Knockout to Probe Single Particle Occupancies in the Ca Isotopes                           & Quantum Monte Carlo calculations with chiral two- and three-nucleon forces \\
\hline
11.30am-2pm                                                                         & Lunch                                                                                             & Lunch                                                                                      & Lunch                                                                                                       & Lunch                                                                                              & Lunch                                                                      \\
\hline
2pm-3.30pm                                                                          & Discussions                                                                                       & Discussions                                                                                & Discussions                                                                                                 & \href{{http://public.lanl.gov/jlynn/}}{Joel Lynn} (LANL)                                           & Discussions                                                                \\
                                                                                    &                                                                                                   &                                                                                            &                                                                                                             & Green's function Monte Carlo calculations of light nuclei with two- and three-nucleon interactions &                                                                            \\
\hline
3.30pm-4pm                                                                          & Coffee break                                                                                      & Coffee break                                                                               & Coffee break                                                                                                & Coffee break                                                                                       & Coffee break                                                               \\
4pm-6pm                                                                             & Discussions                                                                                       & Discussions                                                                                & Discussions                                                                                                 & Discussions                                                                                        & Discussions                                                                \\
\hline
\end{tabular}
}

\noindent
\end{block}
\end{frame}

\begin{frame}[plain,fragile]
\frametitle{Program second week May 18-22 2015}

\begin{block}{}


{\footnotesize
\begin{tabular}{cccccc}
\hline
\multicolumn{1}{c}{ Time } & \multicolumn{1}{c}{ Monday } & \multicolumn{1}{c}{ Tuesday } & \multicolumn{1}{c}{ Wednesday } & \multicolumn{1}{c}{ Thursday } & \multicolumn{1}{c}{ Friday } \\
\hline
9am-10am        & \href{{http://www.chalmers.se/en/staff/Pages/Jimmy-Rotureau.aspx}}{Jimmy Rotureau} (ORNL/MSU)             & \href{{https://www.physics.wustl.edu/people/mahzoon_hossein}}{Hossein Mahzoon} (St Louis) & \href{{http://crunch.ikp.physik.tu-darmstadt.de/tnp/index.php}}{Robert Roth} (Darmstadt)                  & \href{{http://w3.physics.arizona.edu/people/bruce-barrett}}{Bruce Barrett} (UA) & Marcella Grasso (Orsay)                                                                                \\
                & Towards optical potentials from coupled cluster theory                                                    & Dispersive Optical Model                                                                  & News from \emph{ab initio} theory                                                                         & Microscopic shell-model calculations in the $sd$-shell                          & Beyond-mean-field corrections and effective interactions                                               \\
\hline
10am-11am       & \href{{http://www.physastro.iastate.edu/node/8671}}{George Papadimitriou} (ISU)                           & \href{{https://people.nscl.msu.edu/~kohley/}}{Zach Kohley} (MSU)                          & \href{{http://www.triumf.ca/ragnar-stroberg}}{Ragnar Stroberg} (TRIUMF)                                   & \href{{http://www.triumf.ca/angelo-calci}}{Angelo Calci} (TRIUMF)               & \href{{http://irfu.cea.fr/Sphn/Phocea/Membres/Annuaire/index.php?uid=tduguet}}{Thomas Duguet} (Saclay) \\
                & Structure and Reactions of nuclei using complex energy formalisms                                         & Structure and Decay of Neutron Unbound Systems                                            & Ab-initio shell model with IM-SRG                                                                         & SRG Evolved Chiral \emph{NN+3N} Interactions in ab initio Nuclear Structure     & Symmetry restored coupled cluster theory                                                               \\
\hline
11am-11.30am    & Coffee break                                                                                              & Coffee break                                                                              & Coffee break                                                                                              & Coffee break                                                                    & Coffee break                                                                                           \\
\hline
11.30am-12.30pm & \href{{http://www.physics.fsu.edu/people/Personnel.asp?fn=Alexander&ln=Volya&mn=}}{Alexander Volya} (FSU) & \href{{http://nscl.msu.edu/directory/index.php?initial=F}}{Kevin Fossez} (MSU)            & \href{{http://theorie.ikp.physik.tu-darmstadt.de/strongint/people_hebeler.html}}{Kai Hebeler} (Darmstadt) & \href{{http://www.public.iastate.edu/~jvary/}}{James Vary} (ISU)                & \href{{http://www.pa.msu.edu/profile/morri502}}{Titus Morris} (MSU)                                    \\
                & Shell-model approach to nuclear clustering                                                                & Recent progress in the Gamow shell model                                                  & Neutron-rich matter and chiral EFT interactions at N3LO                                                   & Recent advances and new perspectives for the no-core shell model                & In-medium SRG                                                                                          \\
\hline
12.30pm-2pm     & Lunch                                                                                                     & Lunch                                                                                     & Lunch                                                                                                     & Lunch                                                                           & Lunch                                                                                                  \\
\hline
2pm-3.30pm      & Discussions                                                                                               & Discussions                                                                               & Discussions                                                                                               & Discussions                                                                     & Discussions                                                                                            \\
\hline
3.30pm-4pm      & Coffee break                                                                                              & Coffee break                                                                              & Coffee break                                                                                              & Coffee break                                                                    & Coffee break                                                                                           \\
4pm-6pm         & Discussions                                                                                               & Discussions                                                                               & Discussions                                                                                               & Discussions                                                                     & Discussions                                                                                            \\
\hline
\end{tabular}
}

\noindent

\end{block}
\end{frame}

\begin{frame}[plain,fragile]
\frametitle{Program third week May 25-29 2015}

\begin{block}{}


{\footnotesize
\begin{tabular}{cccccc}
\hline
\multicolumn{1}{c}{ Time } & \multicolumn{1}{c}{ Monday } & \multicolumn{1}{c}{ Tuesday } & \multicolumn{1}{c}{ Wednesday } & \multicolumn{1}{c}{ Thursday } & \multicolumn{1}{c}{ Friday } \\
\hline
9am-10am        & \href{{http://people.na.infn.it/~coraggio/}}{Luigi Coraggio} (INFN, Naples)                  & \href{{http://www.cenbg.in2p3.fr/-Group-members,149-?lang=en}}{Nadya Smirnova} (Bordeaux)            & \href{{http://www.physics.sdsu.edu/~johnson/}}{Calvin Johnson} (UCSD)          & \href{{http://www.ph.sophia.ac.jp/~ktak-ken/teacher_en.html}}{Kazuo Takayanagi} (Tokyo) & \href{{https://scholar.google.fr/citations?user=jFwMn0AAAAAJ&hl=en}}{Frederic Nowacki} (Strasbourg) \\
                & A new double-step procedure for the derivation of effective shell-model hamiltonians         & Isospin-symmetry breaking in $sd$ and $pf$ shell nuclei and applications                             & Large scale shell model calculations for open shell nuclei                     & Inverse scattering problem and generalized optical theorem                              & Shell-model far from stability                                                                      \\
\hline
10am-10.30am    & Coffee break                                                                                 & Coffee break                                                                                         & Coffee break                                                                   & Coffee break                                                                            & Coffee break                                                                                        \\
\hline
10.30am-11.30am & \href{{http://nucl.phys.s.u-tokyo.ac.jp/SIR2010/abstract/Suzuki.pdf}}{Toshio Suzuki} (Tokyo) & \href{{http://theory.gsi.de/~ksieja/}}{Kamila Sieja} (Strasbourg)                                    & \href{{http://web.utk.edu/~rgrzywac/}}{Robert Gzrywacz} (ORNL/UTK)             & \href{{http://people.na.infn.it/~itaco/}}{Nunzio Itaco} (Naples)                        & \href{{http://www.researchgate.net/profile/Naofumi_Tsunoda/publications}}{Naofumi Tsunoda} (Tokyo)  \\
                & Evaluation of nuclear weak rates relevant to astrophysical applications                      & Shell evolution and collectivity in model spaces above 78Ni and 132Sn cores                          & Experimental studies of the deeply bound states                                & Present status of perturbative calculation of effective shell-model hamiltonians        & Neutron-rich nuclei from the nuclear force                                                          \\
\hline
11.30am-2pm     & Lunch                                                                                        & Lunch                                                                                                & Lunch                                                                          & Lunch                                                                                   & Lunch                                                                                               \\
\hline
2pm-3.30pm      & Discussions                                                                                  & \href{{http://www.worldscientific.com/doi/abs/10.1142/S0218301308009732}}{Houda Naidja} (Strasbourg) & \href{{http://www.physastro.iastate.edu/directory/pmaris}}{Pieter Maris} (ISU) & Discussions                                                                             & Discussions                                                                                         \\
                &                                                                                              & New advances in nuclear structure studies beyond 132Sn                                               & Neutrons in a trap with chiral interactions                                    &                                                                                         &                                                                                                     \\
\hline
3.30pm-4pm      & Coffee break                                                                                 & Coffee break                                                                                         & Coffee break                                                                   & Coffee break                                                                            & Coffee break                                                                                        \\
4pm-6pm         & Discussions                                                                                  & Discussions                                                                                          & Discussions                                                                    & Discussions                                                                             & Discussions                                                                                         \\
\hline
\end{tabular}
}

\noindent

\end{block}
\end{frame}

\end{document}
